%&"../template"
% The line above indicates the fmt file to use.
\endofdump
%-------------------------------------------------
%     Dynamic premable indicated by \endofdump
%-------------------------------------------------
\tikzexternalize[prefix=cache/]{project1}
\begin{document}
    \title{Report Boost 说明文档}
    \subtitle{Report Boost README}
    \author{Log Creative}
    \maketitle

\section{简明使用方法}
\begin{enumerate}
    \item 使用 VS Code 打开本文件夹。
    \item 使用 \verb"Receipe: recompile header" 编译 \href{run:../template}{template} 以获得静态依赖缓存,以 \verb".fmt" 结尾。
    \item 使用 \verb"Receipe: latexmk" 编译 \href{run:project1}{project1/project1.tex} 即可使用该缓存。文件的第一行表明了需要使用的缓存地址,\verb"\endofdump" 表明了静态依赖库的使用结束位置,之后为你需要为该文档新加入的导言区内容。
\end{enumerate}

\section{路线图}

\begin{itemize}
    \item[$\surd$] 采用 \verb"mylatexformat" 对模板的依赖进行存储,这样可以减少编译时寻找库所花费的时间,并且可以免除以前的局部依赖文档类不在同一个文件夹的问题。
    \item[$\surd$] 兼容于 Ti\emph{k}Z \verb"external" 库用于缓存 Ti\emph{k}Z 图像。
    \item 多线程
    \item xelatex 支持与 CI 集成
    \item 集成脚本
\end{itemize}

\vfill

{
\scriptsize\centering
Licensed under \href{run:../LICENSE}{DO WHAT THE FUCK YOU WANT TO PUBLIC LICENSE}.
}
\end{document}